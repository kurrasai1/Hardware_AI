\documentclass[12pt]{article}
\usepackage[margin=1in]{geometry}
\usepackage{titlesec}
\usepackage{lmodern}
\usepackage{hyperref}
\titleformat{\section}{\normalfont\Large\bfseries}{\thesection}{1em}{}

\title{ECE 410/510 Spring 2025 \\ Codefest Challenge \#4 Report}
\author{Megha Sai Sumanth Kurra}
\date{\05/12/2025}

\begin{document}

\maketitle

\section{Introduction}
This report summarizes the implementation of Challenge \#4 of ECE 410/510 Spring 2025. 
The goal was to replicate the methodology described in the Johns Hopkins paper titled 
`Designing Silicon Brains using LLM'' and to explore LLM-assisted HDL generation for spiking neuron arrays'.

\section{Paper Overview}
The referenced paper leverages large language models like ChatGPT to design digital circuits---specifically 
a spiking neuron array based on the Leaky Integrate-and-Fire (LIF) model. The report outlines how an LLM 
can assist in HDL code generation, testbench creation, and ASIC conversion using OpenLane.

\section{HDL Design Using LLM}
Using ChatGPT, Verilog code for a 4-neuron LIF array was generated. Key parameters such as membrane 
potential, threshold, and leak factor were incorporated. The design was validated through a testbench.

\section{Simulation and Testing}
A testbench module was created to stimulate the neuron array with varying input currents. The design 
was simulated using Icarus Verilog and validated by observing spike outputs and potential resets.

\section{GDS Generation via OpenLane}
Although optional, the design was prepared for ASIC flow. Using OpenLane, the HDL was converted into a GDSII 
layout. This required modifications to the configuration scripts and verified compatibility with OpenLane's 
design flow.

\section{Alternative Neuron Models}
Additional neuron models such as the Rectified Linear Unit (ReLU) and a simplified Hodgkin--Huxley neuron were 
explored. While ReLU was easily implemented in Verilog, the latter was deemed too complex for RTL and recommended 
for high-level modeling.

\section{Results and Comparison}
The HDL design successfully matched the logic described in the Johns Hopkins paper. Functionality, spike timing, 
and behavior were validated. The GDS layout showed acceptable area and cell usage.

\section{Improvement Suggestions}
Suggested improvements include: parameterized neuron counts, better I/O interface (e.g., SPI), pipeline optimization, 
and enhanced ASIC constraints for area and timing.

\section{Documentation and Deliverables}
All LLM queries, Verilog code, simulation logs, and documentation have been uploaded to a public GitHub repository. 
This includes testbench files, waveforms, and configuration scripts.

\end{document}
